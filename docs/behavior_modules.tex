% Options for packages loaded elsewhere
\PassOptionsToPackage{unicode}{hyperref}
\PassOptionsToPackage{hyphens}{url}
%
\documentclass[
]{article}
\usepackage{amsmath,amssymb}
\usepackage{iftex}
\ifPDFTeX
  \usepackage[T1]{fontenc}
  \usepackage[utf8]{inputenc}
  \usepackage{textcomp} % provide euro and other symbols
\else % if luatex or xetex
  \usepackage{unicode-math} % this also loads fontspec
  \defaultfontfeatures{Scale=MatchLowercase}
  \defaultfontfeatures[\rmfamily]{Ligatures=TeX,Scale=1}
\fi
\usepackage{lmodern}
\ifPDFTeX\else
  % xetex/luatex font selection
\fi
% Use upquote if available, for straight quotes in verbatim environments
\IfFileExists{upquote.sty}{\usepackage{upquote}}{}
\IfFileExists{microtype.sty}{% use microtype if available
  \usepackage[]{microtype}
  \UseMicrotypeSet[protrusion]{basicmath} % disable protrusion for tt fonts
}{}
\makeatletter
\@ifundefined{KOMAClassName}{% if non-KOMA class
  \IfFileExists{parskip.sty}{%
    \usepackage{parskip}
  }{% else
    \setlength{\parindent}{0pt}
    \setlength{\parskip}{6pt plus 2pt minus 1pt}}
}{% if KOMA class
  \KOMAoptions{parskip=half}}
\makeatother
\usepackage{xcolor}
\usepackage[margin=1in]{geometry}
\usepackage{color}
\usepackage{fancyvrb}
\newcommand{\VerbBar}{|}
\newcommand{\VERB}{\Verb[commandchars=\\\{\}]}
\DefineVerbatimEnvironment{Highlighting}{Verbatim}{commandchars=\\\{\}}
% Add ',fontsize=\small' for more characters per line
\usepackage{framed}
\definecolor{shadecolor}{RGB}{248,248,248}
\newenvironment{Shaded}{\begin{snugshade}}{\end{snugshade}}
\newcommand{\AlertTok}[1]{\textcolor[rgb]{0.94,0.16,0.16}{#1}}
\newcommand{\AnnotationTok}[1]{\textcolor[rgb]{0.56,0.35,0.01}{\textbf{\textit{#1}}}}
\newcommand{\AttributeTok}[1]{\textcolor[rgb]{0.13,0.29,0.53}{#1}}
\newcommand{\BaseNTok}[1]{\textcolor[rgb]{0.00,0.00,0.81}{#1}}
\newcommand{\BuiltInTok}[1]{#1}
\newcommand{\CharTok}[1]{\textcolor[rgb]{0.31,0.60,0.02}{#1}}
\newcommand{\CommentTok}[1]{\textcolor[rgb]{0.56,0.35,0.01}{\textit{#1}}}
\newcommand{\CommentVarTok}[1]{\textcolor[rgb]{0.56,0.35,0.01}{\textbf{\textit{#1}}}}
\newcommand{\ConstantTok}[1]{\textcolor[rgb]{0.56,0.35,0.01}{#1}}
\newcommand{\ControlFlowTok}[1]{\textcolor[rgb]{0.13,0.29,0.53}{\textbf{#1}}}
\newcommand{\DataTypeTok}[1]{\textcolor[rgb]{0.13,0.29,0.53}{#1}}
\newcommand{\DecValTok}[1]{\textcolor[rgb]{0.00,0.00,0.81}{#1}}
\newcommand{\DocumentationTok}[1]{\textcolor[rgb]{0.56,0.35,0.01}{\textbf{\textit{#1}}}}
\newcommand{\ErrorTok}[1]{\textcolor[rgb]{0.64,0.00,0.00}{\textbf{#1}}}
\newcommand{\ExtensionTok}[1]{#1}
\newcommand{\FloatTok}[1]{\textcolor[rgb]{0.00,0.00,0.81}{#1}}
\newcommand{\FunctionTok}[1]{\textcolor[rgb]{0.13,0.29,0.53}{\textbf{#1}}}
\newcommand{\ImportTok}[1]{#1}
\newcommand{\InformationTok}[1]{\textcolor[rgb]{0.56,0.35,0.01}{\textbf{\textit{#1}}}}
\newcommand{\KeywordTok}[1]{\textcolor[rgb]{0.13,0.29,0.53}{\textbf{#1}}}
\newcommand{\NormalTok}[1]{#1}
\newcommand{\OperatorTok}[1]{\textcolor[rgb]{0.81,0.36,0.00}{\textbf{#1}}}
\newcommand{\OtherTok}[1]{\textcolor[rgb]{0.56,0.35,0.01}{#1}}
\newcommand{\PreprocessorTok}[1]{\textcolor[rgb]{0.56,0.35,0.01}{\textit{#1}}}
\newcommand{\RegionMarkerTok}[1]{#1}
\newcommand{\SpecialCharTok}[1]{\textcolor[rgb]{0.81,0.36,0.00}{\textbf{#1}}}
\newcommand{\SpecialStringTok}[1]{\textcolor[rgb]{0.31,0.60,0.02}{#1}}
\newcommand{\StringTok}[1]{\textcolor[rgb]{0.31,0.60,0.02}{#1}}
\newcommand{\VariableTok}[1]{\textcolor[rgb]{0.00,0.00,0.00}{#1}}
\newcommand{\VerbatimStringTok}[1]{\textcolor[rgb]{0.31,0.60,0.02}{#1}}
\newcommand{\WarningTok}[1]{\textcolor[rgb]{0.56,0.35,0.01}{\textbf{\textit{#1}}}}
\usepackage{longtable,booktabs,array}
\usepackage{calc} % for calculating minipage widths
% Correct order of tables after \paragraph or \subparagraph
\usepackage{etoolbox}
\makeatletter
\patchcmd\longtable{\par}{\if@noskipsec\mbox{}\fi\par}{}{}
\makeatother
% Allow footnotes in longtable head/foot
\IfFileExists{footnotehyper.sty}{\usepackage{footnotehyper}}{\usepackage{footnote}}
\makesavenoteenv{longtable}
\usepackage{graphicx}
\makeatletter
\def\maxwidth{\ifdim\Gin@nat@width>\linewidth\linewidth\else\Gin@nat@width\fi}
\def\maxheight{\ifdim\Gin@nat@height>\textheight\textheight\else\Gin@nat@height\fi}
\makeatother
% Scale images if necessary, so that they will not overflow the page
% margins by default, and it is still possible to overwrite the defaults
% using explicit options in \includegraphics[width, height, ...]{}
\setkeys{Gin}{width=\maxwidth,height=\maxheight,keepaspectratio}
% Set default figure placement to htbp
\makeatletter
\def\fps@figure{htbp}
\makeatother
\setlength{\emergencystretch}{3em} % prevent overfull lines
\providecommand{\tightlist}{%
  \setlength{\itemsep}{0pt}\setlength{\parskip}{0pt}}
\setcounter{secnumdepth}{-\maxdimen} % remove section numbering
\ifLuaTeX
  \usepackage{selnolig}  % disable illegal ligatures
\fi
\IfFileExists{bookmark.sty}{\usepackage{bookmark}}{\usepackage{hyperref}}
\IfFileExists{xurl.sty}{\usepackage{xurl}}{} % add URL line breaks if available
\urlstyle{same}
\hypersetup{
  hidelinks,
  pdfcreator={LaTeX via pandoc}}

\author{}
\date{\vspace{-2.5em}}

\begin{document}

\hypertarget{writing-behavioral-feedback-modules-in-tax-simulator}{%
\section{Writing behavioral feedback modules in
Tax-Simulator}\label{writing-behavioral-feedback-modules-in-tax-simulator}}

Tax-Simulator allows users to write custom behavioral feedback modules.
These modules can be called in Tax-Simulator to simulate micro-level
responses to policy changes --- that is, to account for behavioral
feedback in revenue estimates. This document describes how
Tax-Simulator's behavioral modules work.

Quickly, a bit of terminology up front. \emph{Static} revenue estimates
contain no behavioral feedback: taxpayer behavior is held fixed across
policy scenarios. \emph{Conventional} revenue estimates allow for
certain types of behavioral responses that reflect tax avoidance rather
than a change of economic substance. Examples include business income
shifting across legal entity form and the timing of capital gains
realizations. \emph{Partial dynamic} estimates further expand the range
of incorporated behaviors, allowing for first-order economic changes
such as labor supply changes. \emph{Dynamic}, or full dynamic, refers to
``true'' general equilibrium modeling in which macroeconomic factor
prices like interest rates are allowed to vary. \textbf{Tax-Simulator
can be used to generate conventional and partial dynamic estimates
only.} Full dynamic requires a notion of general equilibrium available
only in computational macro models.

Below, I describe how behavioral feedback modules work and how to write
your own.

\hypertarget{some-architecture-basics}{%
\subsection{Some architecture basics}\label{some-architecture-basics}}

This section describes where behavioral feedback fits into
Tax-Simulator.

\hypertarget{types-of-functions}{%
\subsubsection{Types of functions}\label{types-of-functions}}

Tax-Simulator can be thought of as one big function mapping \emph{input}
attributes of tax microdata (variables like number of children, income,
and expenses) to \emph{output} attributes (variables like deductions,
tax liability, and credits). It is made of many smaller functions. These
smaller functions take one of two forms:

\begin{itemize}
\item
  \textbf{Tax calculation functions.} These functions calculate output
  attributes for an arbitrary set of inputs. For example,
  \texttt{calc\_ctc()} returns the Child Tax Credit for a set of input
  tax units. This is the TurboTax portion of the code.
\item
  \textbf{Economic functions.} These functions act on the input
  attributes of tax microdata.
\end{itemize}

Behavioral feedback modules fall into the latter category. A behavioral
feedback module is, literally, an R function expressing input attributes
(e.g.~capital gains) as a function of tax policy (e.g.~the marginal
effective tax rate on capital gains). When tax policy changes, these
modules are called and update the input attributes of tax microdata to
reflect behavioral feedback.

\hypertarget{static-mode}{%
\subsubsection{Static mode}\label{static-mode}}

At runtime, if told to execute a behavioral feedback module as part of a
simulation of a policy reform, Tax-Simulator will run non-baseline
scenarios twice. First, it runs in \emph{static mode.} This means all
input attributes are held fixed at their baseline levels when
calculating taxes under the policy reform. Critically, this step allows
us to assess the first-order effects of a policy --- for example,
effective marginal tax rates (EMTRs) under the new policy regime.

Then Tax-Simulator runs the policy reform scenario again, this time in
\emph{non-static} mode. \textbf{Behavioral feedback modules are executed
at the beginning of non-static mode.} These modules modify input
attributes according to their specific logic, then Tax-Simulator
re-calculates taxes. The result is a policy reform simulation which
accounts for behavioral feedback.

\hypertarget{organization-and-naming-conventions}{%
\subsubsection{Organization and naming
conventions}\label{organization-and-naming-conventions}}

In the Tax-Simulator repository, code and data are organized into one of
two folders. The \emph{/src} folder contains scenario-invariant model
code --- which is to say, most of it. The \emph{/config} folder contains
configuration files which are used to define tax policy scenarios,
economic assumptions, and other specifications about execution. Despite
being made of code, behavioral feedback modules fall into the
\emph{/config} bucket: these modules \emph{configure} the assumptions
about how taxpayers react to tax reforms. In other words, along with tax
law, behavioral feedback assumptions \emph{define} a scenario.

As such, behavioral feedback modules are .R files stored in
\emph{/config/scenarios/behavior}. Within this directly, modules are
organized into subfolders by the type of behavior modeled.

Let's take \emph{/charity} as an example. This subfolder contains
modules that modify charitable contributions in response to policy
reforms that directly or indirectly affect the tax subsidy for giving.
There are two modules in \emph{/charity}. 100.R simulates a tax-price
elasticity of charitable giving of -1; 50.R assumes a value of -0.5.
Module names should be brief and describe what the module does. We will
look at the contents of these modules below.

Critically, both modules contain a functional called
\texttt{do\_charity()}; ``charity'' in ``do\_charity'' is a keyword
which communicates specific information to Tax-Simulator when it looks
for functions to execute. A behavioral feedback module in the subfolder
\emph{/config/scenarios/behavior/X} must contain a function called
\texttt{do\_X()}.

\hypertarget{configuring-the-model-to-execute-with-specific-behavioral-feedback-modules}{%
\subsubsection{Configuring the model to execute with specific behavioral
feedback
modules}\label{configuring-the-model-to-execute-with-specific-behavioral-feedback-modules}}

A \emph{runscript} is a CSV file that defines scenarios for
Tax-Simulator to run. Runscripts are located in
\emph{/config/runscripts}. Every row represents a scenario to run;
columns reflect parameter values defining each scenario.

One column is called ``behavior''. This is where you list the names of
behavioral feedback modules to run as part of a scenario. Modules are
referred to by their filename relative to the \emph{/behavior}
directory. If running more than one module in a single scenario,
separate the filenames with a single space.

For example, let's say I wanted to run a policy experiment where I
raised ordinary tax rates by 2 percentage points, a policy which
increases the implicit tax subsidy for charitable contributions, and
account for the reforms' impact on contributions. I want to assume an
elasticity of -1 under the functional form contained in the
\emph{/charity/100.R} module. I would specify the following runscript
(extraneous columns are left hidden):

\begin{longtable}[]{@{}
  >{\raggedright\arraybackslash}p{(\columnwidth - 12\tabcolsep) * \real{0.1316}}
  >{\raggedright\arraybackslash}p{(\columnwidth - 12\tabcolsep) * \real{0.0658}}
  >{\raggedright\arraybackslash}p{(\columnwidth - 12\tabcolsep) * \real{0.2632}}
  >{\raggedright\arraybackslash}p{(\columnwidth - 12\tabcolsep) * \real{0.1711}}
  >{\raggedright\arraybackslash}p{(\columnwidth - 12\tabcolsep) * \real{0.0658}}
  >{\raggedright\arraybackslash}p{(\columnwidth - 12\tabcolsep) * \real{0.1447}}
  >{\raggedright\arraybackslash}p{(\columnwidth - 12\tabcolsep) * \real{0.1579}}@{}}
\toprule\noalign{}
\begin{minipage}[b]{\linewidth}\raggedright
ID
\end{minipage} & \begin{minipage}[b]{\linewidth}\raggedright
\ldots{}
\end{minipage} & \begin{minipage}[b]{\linewidth}\raggedright
tax\_law
\end{minipage} & \begin{minipage}[b]{\linewidth}\raggedright
behavior
\end{minipage} & \begin{minipage}[b]{\linewidth}\raggedright
\ldots{}
\end{minipage} & \begin{minipage}[b]{\linewidth}\raggedright
mtr\_vars
\end{minipage} & \begin{minipage}[b]{\linewidth}\raggedright
mtr\_type
\end{minipage} \\
\midrule\noalign{}
\endhead
\bottomrule\noalign{}
\endlastfoot
baseline & \ldots{} & baseline & & \ldots{} & char\_cash & nextdollar \\
2\_pp & \ldots{} & tests/charity/2\_pp & charity/100 & \ldots{} &
char\_cash & nextdollar \\
\end{longtable}

The ``behavior'' attribute is left blank for the baseline scenario,
since behavioral feedback is a concept that applies only to
counterfactual reforms. For the reform scenario, titled ``2\_pp'', I
specify \emph{charity/100} which tells Tax-Simulator to execute the code
contained in the 100.R module when running this scenario. Because the
code contained in 100.R module depends on knowing the marginal tax rate
on cash contributions (variable ``char\_cash'' in the data) current
current law and under the reform, I include \emph{char\_cash} as in
column ``mtr\_vars'' and \emph{nextdollar} in the column ``mtr\_type'',
which tells Tax-Simulator to compute the next-dollar EMTR on whatever
variables are specified. (The other option for ``mtr\_type'' is
\emph{extensive}, in which the average effective tax rate over the
realized value is calculated by setting the value to \$0.) If my
behavioral feedback module did not depend on any EMTR calculations, I
could leave this column blank.

\hypertarget{module-structure}{%
\subsection{Module structure}\label{module-structure}}

A behavioral feedback module contains a ``do'' function, as described
above. This function must have the following signature:

\begin{Shaded}
\begin{Highlighting}[]
\NormalTok{do\_some\_behavior }\OtherTok{=} \ControlFlowTok{function}\NormalTok{(tax\_units, ...) \{ }

  \CommentTok{\# (function contents)}

\NormalTok{\}}
\end{Highlighting}
\end{Shaded}

The parameter \texttt{tax\_units} is the dataframe containing the tax
microdata. It contains all tax records and their attributes, including
all tax law attributes. For example, a record in \texttt{tax\_units}
contains variables like marital status, capital gains, and the top tax
bracket.

The parameter \texttt{…} simply means that any number of other,
arbitrary arguments can be passed to the function. The reason for this
structure is beyond the scope of this how-to guide. But in practice,
your function will also have access to two additional arguments:
\texttt{baseline\_mtrs} and \texttt{static\_mtrs}. These are dataframes,
equal in length to \texttt{tax\_units}, containing marginal tax rate
variables requested at runtime. If your scenario does not calculate
marginal tax rates, these arguments will be of type \texttt{NULL}.

For example, if I submitted ``wages char\_cash'' as variables for which
to calculate marginal tax rates in my runscripts CSV, the
\texttt{baseline\_mtrs} dataframe will look something like this:

\begin{longtable}[]{@{}llll@{}}
\toprule\noalign{}
year & id & mtr\_wages & mtr\_char\_cash \\
\midrule\noalign{}
\endhead
\bottomrule\noalign{}
\endlastfoot
2024 & 1 & 0.24 & 0 \\
2024 & 2 & 0.32 & -0.32 \\
2024 & 3 & 0.408 & -0.408 \\
\ldots{} & \ldots{} & \ldots{} & \ldots{} \\
\end{longtable}

\ldots and \texttt{static\_mtrs} will look identical, just with
(potentially) different values.

Then, the content of the function is completely arbitrary. \textbf{This
is the key idea behind our design of behavioral feedback modules.}
Rather than be confined to expressing behavior feedback as a specific
instance of a single generic elasticity function, users are free to
impose whatever logic you'd like.

That's not to say there are no out-of-the-box helper functions. Many
behavioral feedback functions will simply take the form of applying an
elasticity to a change in EMTRs for all tax units. Rather than having to
re-invent the wheel, users have \texttt{apply\_mtr\_elasticity()}, a
helper function defined in \emph{/src/sim/behavior.R}, at their
disposal. We will look at an example of its use below.

\hypertarget{example-modules}{%
\subsection{Example modules}\label{example-modules}}

Let's make things concrete by reviewing two examples of behavioral
feedback modules.

\hypertarget{charitable-giving}{%
\subsubsection{Charitable giving}\label{charitable-giving}}

By offering an itemized charitable deduction, the tax code subsidizes
donations for those who itemize. There is a large body of empirical
research measuring the responsiveness of charitable activity to the
``tax price'' of giving, i.e.~the after-tax cost to the taxpayer of a
marginal dollar of giving. Let's say we believe the tax price elasticity
of charitable giving -- that is, the percent change in charitable
contributions for a one percent increase in the tax price --- takes a
value of -1. We wish to account the revenue implications of increased
giving, and thus increased deductions, as a result of an increase in
ordinary tax rates (which lower the tax price of giving). We can write a
behavioral feedback module to model this behavior.

One major benefit of microsimulation as a methodological approach is the
ability to precisely measure EMTRs on any activity via computation. By
adding \$1 to some variable Y for each record, re-calculating taxes, and
looking at the change in tax liability, we obtain the \emph{next-dollar}
EMTR on activity Y. For example, if we add \$1 to an itemizer's
charitable contributions, re-calculate their taxes, and their taxes fall
by 37 cents, the next-dollar EMTR on charitable contributions is 0.37
and the tax price of giving is 1 + -0.37 = 0.63. We can calculate this
EMTR for all filers under baseline and under the policy reform then use
this information in our behavioral feedback calculation.

Here is what a behavior feedback module that leverages EMTRs on
charitable contributions looks like:

\begin{Shaded}
\begin{Highlighting}[]
\NormalTok{do\_charity }\OtherTok{=} \ControlFlowTok{function}\NormalTok{(tax\_units, ...) \{ }
  
  \CommentTok{\#{-}{-}{-}{-}{-}{-}{-}{-}{-}{-}{-}{-}{-}{-}{-}{-}{-}{-}{-}{-}{-}{-}{-}{-}{-}{-}{-}{-}{-}{-}{-}{-}{-}{-}{-}{-}{-}{-}{-}{-}{-}{-}{-}{-}{-}{-}{-}{-}{-}{-}{-}{-}{-}{-}{-}{-}{-}{-}{-}{-}{-}{-}{-}{-}{-}{-}{-}{-}{-}{-}{-}{-}{-}{-}{-}{-}}
  \CommentTok{\# Adjusts cash charitable contributions along the intensive margin with a }
  \CommentTok{\# tax price elasticity of {-}1.}
  \CommentTok{\# }
  \CommentTok{\# Parameters: }
  \CommentTok{\#   {-} tax\_units (df)     : tibble of tax units with calculated variables}
  \CommentTok{\#   {-} baseline\_mtrs (df) : year{-}id indexed tibble of MTRs under the baseline}
  \CommentTok{\#   {-} static\_mtrs (df)   : year{-}id indexed tibble of MTRs under the static}
  \CommentTok{\#                          counterfactual scenario}
  \CommentTok{\#}
  \CommentTok{\# Returns: tibble of tax units with post{-}adjustment cash charitable }
  \CommentTok{\#          contribution values. }
  \CommentTok{\#{-}{-}{-}{-}{-}{-}{-}{-}{-}{-}{-}{-}{-}{-}{-}{-}{-}{-}{-}{-}{-}{-}{-}{-}{-}{-}{-}{-}{-}{-}{-}{-}{-}{-}{-}{-}{-}{-}{-}{-}{-}{-}{-}{-}{-}{-}{-}{-}{-}{-}{-}{-}{-}{-}{-}{-}{-}{-}{-}{-}{-}{-}{-}{-}{-}{-}{-}{-}{-}{-}{-}{-}{-}{-}{-}{-}}
  
  \CommentTok{\# Set elasticity}
\NormalTok{  e }\OtherTok{=} \SpecialCharTok{{-}}\DecValTok{1}
  
  \CommentTok{\# Apply elasticities and calculate new values}
\NormalTok{  new\_values }\OtherTok{=}\NormalTok{ tax\_units }\SpecialCharTok{\%\textgreater{}\%} 
    \FunctionTok{mutate}\NormalTok{(}\AttributeTok{e\_char\_cash      =}\NormalTok{ e, }
           \AttributeTok{e\_char\_cash\_type =} \StringTok{\textquotesingle{}taxprice\textquotesingle{}}\NormalTok{) }\SpecialCharTok{\%\textgreater{}\%} 
    \FunctionTok{apply\_mtr\_elasticity}\NormalTok{(}\StringTok{\textquotesingle{}char\_cash\textquotesingle{}}\NormalTok{, baseline\_mtrs, static\_mtrs, }\DecValTok{1}\NormalTok{)}

  \CommentTok{\# Replace old values with new and return}
\NormalTok{  tax\_units }\SpecialCharTok{\%\textgreater{}\%} 
    \FunctionTok{select}\NormalTok{(}\SpecialCharTok{{-}}\NormalTok{char\_cash) }\SpecialCharTok{\%\textgreater{}\%} 
    \FunctionTok{bind\_cols}\NormalTok{(new\_values) }\SpecialCharTok{\%\textgreater{}\%} 
    \FunctionTok{return}\NormalTok{()}
\NormalTok{\}}
\end{Highlighting}
\end{Shaded}

First: note the formatted function documentation. We require the author
of a module follow the style laid out above. The idea is to describe in
plain English which parameters the module depends on, what it does, and
what it returns. It helps the reader orient themselves before diving
into the code.

The first thing we do is define a variable \texttt{e\ =\ -1} to be our
elasticity. Then we assign this variable, as well as a string indicating
its type, to each record in a copy of \texttt{tax\_units}. Next we feed
the result into a function called \texttt{apply\_mtr\_elasticity()}.
This is a helper function that adjusts a specific variable in a
dataframe based on specified elasticity information, returning a
one-column dataframe of the post-adjustment variable. For a full look at
this function, please see the \emph{/src/sim/behavior.R} file. But
here's a rundown on what each argument in our function call does:

\begin{itemize}
\item
  \texttt{tax\_units\ =\ (.)} \ldots{} This argument, omitted by
  convention in the dplyr chain, is the dataframe of tax units.
  Crucially, it contains three columns with specific names:
  ``char\_cash'', ``e\_char\_cash'', and ``e\_char\_cash\_type''. These
  are required because the next argument is\ldots{}
\item
  \texttt{var\ =\ \textquotesingle{}char\_cash\textquotesingle{}}
  \ldots{} This argument communicates that we wish to adjust the
  variable called ``char\_cash''. It also tells the function that there
  are two associated columns in the \texttt{tax\_units} dataframe:
  ``e\_char\_cash'', which is a filer-level elasticity with respect to
  ``char\_cash'', and ``e\_char\_cash\_type'', which describes the
  functional form of the elasticity. Earlier in the module, we set
  \texttt{e\_char\_cash\ =\ -1} and
  \texttt{e\_char\_cash\_type\ =\ \textquotesingle{}taxprice\textquotesingle{}}.
  Here, ``taxprice'' is a keyword. There are four possible options for
  elasticity type:

  \begin{itemize}
  \item
    ``semi'': a log-lin semi-elasticity, i.e.~it gives percent change in
    Y for a percentage point change in the EMTR.
  \item
    ``arc'': a full log-log elasticity evaluated at the midpoint,
    i.e.~it gives percent change in Y for a percent change in EMTR,
    where the latter is calculated at the midpoint of the two EMTRs.
  \item
    ``netoftax'': a full log-log elasticity on 1 minus the EMTR.
  \item
    ``taxprice'': a full log-log elasticity on 1 plus the EMTR.
  \end{itemize}
\end{itemize}

Here are the actual calculations for each option:

\begin{Shaded}
\begin{Highlighting}[]
\NormalTok{      pct\_chg }\OtherTok{=} \FunctionTok{case\_when}\NormalTok{(}
\NormalTok{        e\_type }\SpecialCharTok{==} \StringTok{"semi"}     \SpecialCharTok{\textasciitilde{}} \FunctionTok{exp}\NormalTok{((mtr }\SpecialCharTok{{-}}\NormalTok{ mtr\_baseline) }\SpecialCharTok{*}\NormalTok{ e) }\SpecialCharTok{{-}} \DecValTok{1}\NormalTok{,}
\NormalTok{        e\_type }\SpecialCharTok{==} \StringTok{"arc"}      \SpecialCharTok{\textasciitilde{}}\NormalTok{ (e }\SpecialCharTok{*}\NormalTok{ (mtr }\SpecialCharTok{/}\NormalTok{ ((mtr }\SpecialCharTok{+}\NormalTok{ mtr\_baseline) }\SpecialCharTok{/} \DecValTok{2}\NormalTok{) }\SpecialCharTok{{-}} \DecValTok{1}\NormalTok{)),}
\NormalTok{        e\_type }\SpecialCharTok{==} \StringTok{"netoftax"} \SpecialCharTok{\textasciitilde{}}\NormalTok{ (e }\SpecialCharTok{*}\NormalTok{ ((}\DecValTok{1} \SpecialCharTok{{-}}\NormalTok{ mtr) }\SpecialCharTok{/}\NormalTok{ (}\DecValTok{1} \SpecialCharTok{{-}}\NormalTok{ mtr\_baseline) }\SpecialCharTok{{-}} \DecValTok{1}\NormalTok{)),}
\NormalTok{        e\_type }\SpecialCharTok{==} \StringTok{"taxprice"} \SpecialCharTok{\textasciitilde{}}\NormalTok{ (e }\SpecialCharTok{*}\NormalTok{ ((}\DecValTok{1} \SpecialCharTok{+}\NormalTok{ mtr) }\SpecialCharTok{/}\NormalTok{ (}\DecValTok{1} \SpecialCharTok{+}\NormalTok{ mtr\_baseline) }\SpecialCharTok{{-}} \DecValTok{1}\NormalTok{)),}
        \ConstantTok{TRUE}                 \SpecialCharTok{\textasciitilde{}} \ConstantTok{NA} 
\NormalTok{      ) }
\end{Highlighting}
\end{Shaded}

\begin{itemize}
\item
  \texttt{baseline\_mtrs\ =\ baseline\_mtrs} \ldots{} This argument
  supplies a dataframe of baseline EMTRs, which in this case must
  contain a variable called ``mtr\_char\_cash''.
\item
  \texttt{static\_mtrs\ =\ static\_mtrs} \ldots{} This argument supplies
  a dataframe of EMTRs under the static simulation of the policy reform,
  which again in this case must contain a variable called
  ``mtr\_char\_cash''.
\item
  \texttt{max\_adj\ =\ 1} \ldots{} This argument limits the absolute
  value of the resulting percent change for any record to 100\%. It
  prevents unreasonable adjustments resulting from edge cases in EMTRs,
  which in rare cases can be extreme when a taxpayer is stuck at a notch
  in the code.
\end{itemize}

To summarize, we set the tax-price charitable contributions elasticity
to -1 for each record. Then we pass our data, including EMTRs on
charitable contributions under the baseline and under the policy reform,
to a helper function. This function multiplies the result of the
following calculation to cash charitable contributions for each
record\ldots{}

\[
1+e \left( \frac{1+EMTR_{policy}}{1+EMTR_{baseline}} \right)
\]

\ldots and returns the resulting new values of cash charitable
contributions. We assign these values to an intermediate dataframe
called \texttt{new\_values}. Finally, in the next block of code, we
replace the old values of cash charitable contributions with these new
values. (Note that we return the whole tax units dataframe, not just the
new column. The result of a behavioral feedback module should always be
the identical \texttt{tax\_units} dataframe input except with new values
for the variable(s) we are simulating feedback for.)

And we're done! The module has applied the logic in our behavioral
feedback module to the simulated tax records. Tax-Simulator will go on
to calculate taxes based on this adjusted data, and our end results will
reflect our assumptions about behavioral responses.

One last note on EMTR elasticity-based behavioral feedback modules. This
example was a comparatively crude example: We are applying the same
elasticity to all tax units in all years. In real life, elasticities may
vary with observable attributes like income (rich people are more
concerned with tax optimization). We could just as easily have assigned
specific elasticities to different types of records in the
\texttt{mutate()} call, and the logic afterwards would be unchanged.

\hypertarget{employment}{%
\subsubsection{Employment}\label{employment}}

The prior module assumed that behavioral responses occur only at the
intensive margin: charitable giving was increased or decreased only
among those who already gave under the baseline. But many behavioral
responses we wish to model are extensive-margin phenomena, in which
nonzero values can become zeros and vice versa. Low-income employment
and its relationship to tax policy is one such example.

In 2021, the Child Tax Credit (CTC) was expanded such that the credit no
longer phased in with earnings --- meaning that the reform increased
EMTRs on low-income workers. This reduction in the ``return to work''
may, via substitution effects, cause some employment loss among
marginally attached low-income workers. A partial dynamic budget cost
estimate will reflect the budgetary impact of this change in employment,
which should reduce payroll taxes and income taxes. We can write a
behavioral feedback module to account for this effect in our cost
estimate.

Below is a behavioral feedback module implementing the logic and
parameters of
\href{file:///C:/Users/jar335/Downloads/Bastian_CTCexpansion_2023.pdf}{Bastian
(2023)}, a paper which estimates employment loss caused by the 2021 CTC
expansion. Bastian calculates a ``return-to-work'' metric, defined as 1
minus the average effective tax rate on working. By demographic group,
he calculates the change in return-to-work caused by the CTC reform and
then applies demographic-specific labor supply elasticities to obtain
employment loss estimates.

\begin{Shaded}
\begin{Highlighting}[]
\NormalTok{do\_employment }\OtherTok{=} \ControlFlowTok{function}\NormalTok{(tax\_units, ...) \{ }
  
  \CommentTok{\#{-}{-}{-}{-}{-}{-}{-}{-}{-}{-}{-}{-}{-}{-}{-}{-}{-}{-}{-}{-}{-}{-}{-}{-}{-}{-}{-}{-}{-}{-}{-}{-}{-}{-}{-}{-}{-}{-}{-}{-}{-}{-}{-}{-}{-}{-}{-}{-}{-}{-}{-}{-}{-}{-}{-}{-}{-}{-}{-}{-}{-}{-}{-}{-}{-}{-}{-}{-}{-}{-}{-}{-}{-}{-}{-}{-}}
  \CommentTok{\# Adjusts wage earnings at the extensive margin, per Bastian (2023). Only}
  \CommentTok{\# suitable to analyze an *increase* in EMTRs at the low end {-}{-} there is no}
  \CommentTok{\# symmetric effect in which nonworkers become workers in response to a cut in}
  \CommentTok{\# EMTRs. Used in our analysis of the employment effects of the 2021 CTC. }
  \CommentTok{\# }
  \CommentTok{\# Parameters: }
  \CommentTok{\#   {-} tax\_units (df)     : tibble of tax units with calculated variables}
  \CommentTok{\#   {-} baseline\_mtrs (df) : year{-}id indexed tibble of extensive{-}margin MTRs on }
  \CommentTok{\#                          wages1 and wages2 under the baseline}
  \CommentTok{\#   {-} static\_mtrs (df)   : year{-}id indexed tibble of extensive{-}margin MTRs on }
  \CommentTok{\#                          wages1 and wages2 under the static counterfactual}
  \CommentTok{\#}
  \CommentTok{\# Returns: tibble of tax units with post{-}adjustment wage earnings values.}
  \CommentTok{\#{-}{-}{-}{-}{-}{-}{-}{-}{-}{-}{-}{-}{-}{-}{-}{-}{-}{-}{-}{-}{-}{-}{-}{-}{-}{-}{-}{-}{-}{-}{-}{-}{-}{-}{-}{-}{-}{-}{-}{-}{-}{-}{-}{-}{-}{-}{-}{-}{-}{-}{-}{-}{-}{-}{-}{-}{-}{-}{-}{-}{-}{-}{-}{-}{-}{-}{-}{-}{-}{-}{-}{-}{-}{-}{-}{-}}
  
  \CommentTok{\# Set random seed }
  \FunctionTok{set.seed}\NormalTok{(globals}\SpecialCharTok{$}\NormalTok{random\_seed)}
  
  \CommentTok{\# Set elasticities}
\NormalTok{  e\_mothers\_poor  }\OtherTok{=} \FloatTok{0.4}
\NormalTok{  e\_mothers\_other }\OtherTok{=} \FloatTok{0.2}
\NormalTok{  e\_else          }\OtherTok{=} \FloatTok{0.05}
  
  
\NormalTok{  tax\_units }\SpecialCharTok{\%\textgreater{}\%} 
    
    \CommentTok{\# Join MTRs}
    \FunctionTok{left\_join}\NormalTok{(baseline\_mtrs }\SpecialCharTok{\%\textgreater{}\%} 
                \FunctionTok{rename\_with}\NormalTok{(}\AttributeTok{.cols =} \SpecialCharTok{{-}}\FunctionTok{c}\NormalTok{(id, year), }
                            \AttributeTok{.fn   =} \SpecialCharTok{\textasciitilde{}} \FunctionTok{paste0}\NormalTok{(., }\StringTok{\textquotesingle{}\_baseline\textquotesingle{}}\NormalTok{)), }
              \AttributeTok{by =} \FunctionTok{c}\NormalTok{(}\StringTok{\textquotesingle{}id\textquotesingle{}}\NormalTok{, }\StringTok{\textquotesingle{}year\textquotesingle{}}\NormalTok{)) }\SpecialCharTok{\%\textgreater{}\%}
    \FunctionTok{left\_join}\NormalTok{(static\_mtrs, }\AttributeTok{by =} \FunctionTok{c}\NormalTok{(}\StringTok{\textquotesingle{}id\textquotesingle{}}\NormalTok{, }\StringTok{\textquotesingle{}year\textquotesingle{}}\NormalTok{)) }\SpecialCharTok{\%\textgreater{}\%} 
    
    \FunctionTok{mutate}\NormalTok{(}

      \CommentTok{\# Calculate tax unit{-}level income (roughly AGI)}
      \AttributeTok{income =}\NormalTok{ wages }\SpecialCharTok{+}\NormalTok{ txbl\_int }\SpecialCharTok{+}\NormalTok{ div\_ord }\SpecialCharTok{+}\NormalTok{ div\_pref }\SpecialCharTok{+}\NormalTok{ state\_ref }\SpecialCharTok{+} 
\NormalTok{               txbl\_ira\_dist }\SpecialCharTok{+}\NormalTok{ txbl\_pens\_dist }\SpecialCharTok{+}\NormalTok{ kg\_lt }\SpecialCharTok{+}\NormalTok{ kg\_st }\SpecialCharTok{+}\NormalTok{ other\_gains }\SpecialCharTok{+} 
\NormalTok{               sole\_prop }\SpecialCharTok{+}\NormalTok{ part\_active }\SpecialCharTok{+}\NormalTok{ part\_passive }\SpecialCharTok{{-}}\NormalTok{ part\_active\_loss }\SpecialCharTok{{-}} 
\NormalTok{               part\_passive\_loss }\SpecialCharTok{{-}}\NormalTok{ part\_179 }\SpecialCharTok{+}\NormalTok{ scorp }\SpecialCharTok{+}\NormalTok{ scorp\_active }\SpecialCharTok{+} 
\NormalTok{               scorp\_passive }\SpecialCharTok{{-}}\NormalTok{ scorp\_active\_loss }\SpecialCharTok{{-}}\NormalTok{ scorp\_passive\_loss }\SpecialCharTok{{-}} 
\NormalTok{               scorp\_179 }\SpecialCharTok{+}\NormalTok{ rent }\SpecialCharTok{{-}}\NormalTok{ rent\_loss }\SpecialCharTok{+}\NormalTok{ estate }\SpecialCharTok{{-}}\NormalTok{ estate\_loss }\SpecialCharTok{+}\NormalTok{ farm }\SpecialCharTok{+}\NormalTok{ ui }\SpecialCharTok{+} 
\NormalTok{               gross\_ss }\SpecialCharTok{+}\NormalTok{ other\_inc,}
      
      \CommentTok{\#{-}{-}{-}{-}{-}{-}{-}{-}{-}{-}{-}{-}{-}{-}{-}{-}{-}{-}}
      \CommentTok{\# Set elasticities}
      \CommentTok{\#{-}{-}{-}{-}{-}{-}{-}{-}{-}{-}{-}{-}{-}{-}{-}{-}{-}{-}}
      
      \CommentTok{\# First earner}
      \AttributeTok{e1 =} \FunctionTok{case\_when}\NormalTok{(}
        
        \CommentTok{\# Low{-}income single mothers}
\NormalTok{        (male1 }\SpecialCharTok{==} \DecValTok{0}\NormalTok{) }\SpecialCharTok{\&}\NormalTok{ (n\_dep\_ctc }\SpecialCharTok{\textgreater{}} \DecValTok{0}\NormalTok{) }\SpecialCharTok{\&}\NormalTok{ (wages1 }\SpecialCharTok{\textless{}}\NormalTok{ eitc.po\_thresh\_1) }\SpecialCharTok{\&}\NormalTok{ (filing\_status }\SpecialCharTok{!=} \DecValTok{2}\NormalTok{) }\SpecialCharTok{\textasciitilde{}}\NormalTok{ e\_mothers\_poor, }
        
        \CommentTok{\# All other mothers with family income below $80,000 }
\NormalTok{        (male1 }\SpecialCharTok{==} \DecValTok{0}\NormalTok{) }\SpecialCharTok{\&}\NormalTok{ (n\_dep\_ctc }\SpecialCharTok{\textgreater{}} \DecValTok{0}\NormalTok{) }\SpecialCharTok{\&}\NormalTok{ (income }\SpecialCharTok{\textless{}} \DecValTok{80000}\NormalTok{) }\SpecialCharTok{\textasciitilde{}}\NormalTok{ e\_mothers\_other, }
        
        \CommentTok{\# Others below $80,000}
\NormalTok{        (income }\SpecialCharTok{\textless{}} \DecValTok{80000} \SpecialCharTok{\&}\NormalTok{ n\_dep\_ctc }\SpecialCharTok{\textgreater{}} \DecValTok{0}\NormalTok{) }\SpecialCharTok{\textasciitilde{}}\NormalTok{ e\_else,}
        
        \CommentTok{\# Everyone else}
        \ConstantTok{TRUE} \SpecialCharTok{\textasciitilde{}} \DecValTok{0}
\NormalTok{      ),}
      
      \CommentTok{\# Second earner}
      \AttributeTok{e2 =} \FunctionTok{case\_when}\NormalTok{(}
        
        \CommentTok{\# Low{-}income single mothers}
\NormalTok{        (male1 }\SpecialCharTok{==} \DecValTok{0}\NormalTok{) }\SpecialCharTok{\&}\NormalTok{ (n\_dep\_ctc }\SpecialCharTok{\textgreater{}} \DecValTok{0}\NormalTok{) }\SpecialCharTok{\&}\NormalTok{ (wages1 }\SpecialCharTok{\textless{}}\NormalTok{ eitc.po\_thresh\_1) }\SpecialCharTok{\&}\NormalTok{ (filing\_status }\SpecialCharTok{!=} \DecValTok{2}\NormalTok{) }\SpecialCharTok{\textasciitilde{}}\NormalTok{ e\_mothers\_poor, }
        
        \CommentTok{\# All other mothers with family income below $80,000 }
\NormalTok{        (male1 }\SpecialCharTok{==} \DecValTok{0}\NormalTok{) }\SpecialCharTok{\&}\NormalTok{ (n\_dep\_ctc }\SpecialCharTok{\textgreater{}} \DecValTok{0}\NormalTok{) }\SpecialCharTok{\&}\NormalTok{ (income }\SpecialCharTok{\textless{}} \DecValTok{80000}\NormalTok{) }\SpecialCharTok{\textasciitilde{}}\NormalTok{ e\_mothers\_other, }
        
        \CommentTok{\# Others below $80,000}
\NormalTok{        (income }\SpecialCharTok{\textless{}} \DecValTok{80000} \SpecialCharTok{\&}\NormalTok{ n\_dep\_ctc }\SpecialCharTok{\textgreater{}} \DecValTok{0}\NormalTok{) }\SpecialCharTok{\textasciitilde{}}\NormalTok{ e\_else,}
        
        \CommentTok{\# Everyone else}
        \ConstantTok{TRUE} \SpecialCharTok{\textasciitilde{}} \DecValTok{0}
\NormalTok{      ),}
      
      
      \CommentTok{\#{-}{-}{-}{-}{-}{-}{-}{-}{-}{-}{-}{-}{-}{-}{-}{-}{-}{-}{-}{-}{-}{-}{-}{-}{-}{-}{-}}
      \CommentTok{\# Simulate labor force exit}
      \CommentTok{\#{-}{-}{-}{-}{-}{-}{-}{-}{-}{-}{-}{-}{-}{-}{-}{-}{-}{-}{-}{-}{-}{-}{-}{-}{-}{-}{-}}
      
      \CommentTok{\# Calculate percent change in return{-}to{-}work}
      \AttributeTok{delta\_rtw1 =}\NormalTok{ ((}\DecValTok{1} \SpecialCharTok{{-}}\NormalTok{ mtr\_wages1) }\SpecialCharTok{{-}}\NormalTok{ (}\DecValTok{1} \SpecialCharTok{{-}}\NormalTok{ mtr\_wages1\_baseline)) }\SpecialCharTok{/}\NormalTok{ (}\DecValTok{1} \SpecialCharTok{{-}}\NormalTok{ mtr\_wages1\_baseline),}
      \AttributeTok{delta\_rtw2 =}\NormalTok{ ((}\DecValTok{1} \SpecialCharTok{{-}}\NormalTok{ mtr\_wages2) }\SpecialCharTok{{-}}\NormalTok{ (}\DecValTok{1} \SpecialCharTok{{-}}\NormalTok{ mtr\_wages2\_baseline)) }\SpecialCharTok{/}\NormalTok{ (}\DecValTok{1} \SpecialCharTok{{-}}\NormalTok{ mtr\_wages2\_baseline),}
      
      \CommentTok{\# Calculate probability of remaining employed defined as 1 plus the }
      \CommentTok{\# implied percent change in employment }
      \AttributeTok{pr\_emp1 =} \DecValTok{1} \SpecialCharTok{+}\NormalTok{ (e1 }\SpecialCharTok{*}\NormalTok{ delta\_rtw1),}
      \AttributeTok{pr\_emp2 =} \DecValTok{1} \SpecialCharTok{+}\NormalTok{ (e2 }\SpecialCharTok{*}\NormalTok{ delta\_rtw2),}
      
      \CommentTok{\# Simulate outcomes}
      \AttributeTok{emp1 =} \FunctionTok{runif}\NormalTok{(}\FunctionTok{nrow}\NormalTok{(.)) }\SpecialCharTok{\textless{}}\NormalTok{ pr\_emp1,}
      \AttributeTok{emp2 =} \FunctionTok{runif}\NormalTok{(}\FunctionTok{nrow}\NormalTok{(.)) }\SpecialCharTok{\textless{}}\NormalTok{ pr\_emp2,}
      
      \CommentTok{\# Adjust wages}
      \AttributeTok{wages1 =} \FunctionTok{if\_else}\NormalTok{(wages1 }\SpecialCharTok{==} \DecValTok{0}\NormalTok{, }\DecValTok{0}\NormalTok{, wages1 }\SpecialCharTok{*}\NormalTok{ emp1),}
      \AttributeTok{wages2 =} \FunctionTok{if\_else}\NormalTok{(wages2 }\SpecialCharTok{==} \DecValTok{0}\NormalTok{, }\DecValTok{0}\NormalTok{, wages2 }\SpecialCharTok{*}\NormalTok{ emp2),}
      \AttributeTok{wages  =}\NormalTok{ wages1 }\SpecialCharTok{+}\NormalTok{ wages2 }
      
\NormalTok{    ) }\SpecialCharTok{\%\textgreater{}\%} 
    
    \CommentTok{\# Remove intermediate calculation variables and return }
    \FunctionTok{select}\NormalTok{(}\SpecialCharTok{{-}}\NormalTok{income, }\SpecialCharTok{{-}}\NormalTok{e1, }\SpecialCharTok{{-}}\NormalTok{e2, }\SpecialCharTok{{-}}\NormalTok{delta\_rtw1, }\SpecialCharTok{{-}}\NormalTok{delta\_rtw2, }
           \SpecialCharTok{{-}}\NormalTok{pr\_emp1, }\SpecialCharTok{{-}}\NormalTok{pre\_emp2, }\SpecialCharTok{{-}}\NormalTok{emp1, }\SpecialCharTok{{-}}\NormalTok{emp2) }\SpecialCharTok{\%\textgreater{}\%} 
    \FunctionTok{return}\NormalTok{()}
\NormalTok{\}}
\end{Highlighting}
\end{Shaded}

In terms of function inputs, this module uses extensive-margin tax rates
--- i.e., average effective tax rates --- on primary-earner wages
(``wages1'') and, in the case of joint returns, secondary-earner wages
(``wages2''). These variables are calculated earlier in Tax-Simulator's
execution and are passed to our function in \texttt{baseline\_mtrs} and
\texttt{static\_mtrs}.

At the top, because this module involves stochastic simulation, we reset
the random number generator seed --- a requirement in Tax-Simulator
before any call to a random number generator. We also list our
elasticity assumptions at the top per convention.

Then we join our EMTR dataframes into tax units, making sure to
distinguish baseline from policy reform variables by renaming the
former.

Next, we assign labor supply elasticities to each nondependent adult. As
per Tax-Simulator naming convention, primary earner variables are
appended with ``1'' and secondary variables are appended with ``2''.
Before assignment, we calculate a helper variable, income, since
elasticities are heterogeneous by demographic group and income. The
\texttt{case\_when()} function calls implement Bastian's logic:
EITC-qualifying single mothers have an elasticity of 0.4; other mothers
with income below \$80K have an elasticity of 0.2; others below \$80K
have an elasticity of 0.05; those above \$80K are assumed not to be
responsive. The important takeaway from these lines is that we can build
highly detailed heterogeneity into our behavioral feedback assumptions.

The next step is to calculate the policy reforms's effect on the return
to work, again defined as 1 minus the average effective tax rate
(somewhat confusingly stored with an ``mtr'' prefix per Tax-Simulator
convention). We calculate the percent change in this metric for primary
and secondary earners.

Applying the labor supply elasticity to this quantity gets us the
percent change in employment, which is uninterpretable at the micro
level. Instead, we convert the result into a \emph{probability of
remaining employed} by adding 1. For example, if the reform reduces
someone's return to work by 10\%, and we assume a 0.4 labor supply
elasticity, we'd say that the percent change in employment is -4\%, and
thus the probability of remaining employed is 96\%,
i.e.~\[1 + (-0.1 * 0.4) = 1 - 0.04 = 0.96\].

At this point, we have employment probabilities for all workers in our
data. The final step is to simulate labor force exit using a random
number generator. Wages are set to 0 for those who drop out of the labor
force, and \emph{viola} --- we're done.

\end{document}
